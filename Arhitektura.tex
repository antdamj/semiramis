\chapter{Arhitektura i dizajn sustava}
		
		\textbf{\textit{dio 1. revizije}}\\

		\textit{ Potrebno je opisati stil arhitekture te identificirati: podsustave, preslikavanje na radnu platformu, spremišta podataka, mrežne protokole, globalni upravljački tok i sklopovsko-programske zahtjeve. Po točkama razraditi i popratiti odgovarajućim skicama:}
	\begin{itemize}
		\item 	\textit{izbor arhitekture temeljem principa oblikovanja pokazanih na predavanjima (objasniti zašto ste baš odabrali takvu arhitekturu)}
		\item 	\textit{organizaciju sustava s najviše razine apstrakcije (npr. klijent-poslužitelj, baza podataka, datotečni sustav, grafičko sučelje)}
		\item 	\textit{organizaciju aplikacije (npr. slojevi frontend i backend, MVC arhitektura) }		
	\end{itemize}

	
		

		

				
		\section{Baza podataka}
			
			\textbf{\textit{dio 1. revizije}}\\
			
		\textit{Potrebno je opisati koju vrstu i implementaciju baze podataka ste odabrali, glavne komponente od kojih se sastoji i slično.}
		
		    \noindent Za potrebe našeg sustava koristit ćemo relacijsku bazu podataka koja
            svojom strukturom olakšava modeliranje stvarnog svijeta. Gradivna jedinka
            baze je relacija, odnosno tablica koja je definirana svojim imenom i 
            skupom atributa. Zadaća baze podataka je brza i jednostavna pohrana, 
            izmjena i dohvat podataka za daljnju obradu.
            Baza podataka ove aplikacije sastoji se od sljedećih entiteta:
            
            \begin{packed_item}
                \item Korisnik
                \item Poslovnica
                \item Recenzija
                \item Rezervacija
                \item Session
                \item Vozilo
            \end{packed_item}
		
			\subsection{Opis tablica}
			

				\textit{Svaku tablicu je potrebno opisati po zadanom predlošku. Lijevo se nalazi točno ime varijable u bazi podataka, u sredini se nalazi tip podataka, a desno se nalazi opis varijable. Svjetlozelenom bojom označite primarni ključ. Svjetlo plavom označite strani ključ}
				
				\begin{longtabu} to \textwidth {|X[6, l]|X[6, l]|X[20, l]|}
					
					\hline \multicolumn{3}{|c|}{\textbf{korisnik - ime tablice}}	 \\[3pt] \hline
					\endfirsthead
					
					\hline \multicolumn{3}{|c|}{\textbf{korisnik - ime tablice}}	 \\[3pt] \hline
					\endhead
					
					\hline 
					\endlastfoot
					
					\cellcolor{LightGreen}IDKorisnik & INT	&  	Quo usque tandem abutere, Catilina, patientia nostra? quam diu etiam furor iste tuus nos eludet? quem ad finem sese effrenata iactabit audacia tua? Nihilne te nocturnum praesidium Palati, nihil urbis vigiliae, nihil timor populi, nihil concursus bonorum omnium, nihil hic munitissimus habendi senatus locus, nihil horum ora voltusque moverunt? 	\\ \hline
					korisnickoIme	& VARCHAR &   Patere tua consilia non sentis, constrictam iam horum omnium scientia teneri coniurationem tuam non vides? Quid proxima, quid superiore nocte egeris, ubi fueris, quos convocaveris, quid consilii ceperis, quem nostrum ignorare arbitraris?\\ \hline 
					email & VARCHAR &   \\ \hline 
					ime & VARCHAR	&  		\\ \hline 
					\cellcolor{LightBlue} primjer	& VARCHAR &   	\\ \hline 
					
					
				\end{longtabu}
				
				\noindent \textbf{Korisnik} \quad Ovaj entitet sadržava sve važne informacije o korisniku aplikacije. Sadrži atribute: Korisničko ime, ime, prezime, email, lozinku, broj mobitela korisnika i njegovu ulogu. Ovaj entitet u vezi je
                \textit{One-to-Many} s entitetom Recenzija preko atributa korisničko ime korisnika i u vezi \textit{One-to-Many} s entitetom Rezervacija preko korisničkog imena.
                
                \begin{longtabu} to \textwidth {|X[6, l]|X[6, l]|X[20, l]|}
					
					\hline \multicolumn{3}{|c|}{\textbf{Korisnik}}	 \\[3pt] \hline
					\endfirsthead
					
					\hline \multicolumn{3}{|c|}{\textbf{Korisnik}}	 \\[3pt] \hline
					\endhead
					
					\hline 
					\endlastfoot
					
					\cellcolor{LightGreen}Korisničko ime & VARCHAR	&  	korisničko ime korisnika koje je ujedno i jedinstveni identifikator\\ \hline
					Ime	& VARCHAR &   ime korisnika\\ \hline
					Prezime	& VARCHAR &   prezime korisnika\\ \hline
					Email & VARCHAR &   e-mail adresa korisnika\\ \hline
					Lozinka	& VARCHAR &   lozinka korisnika\\ \hline
					Broj mobitela	& VARCHAR &   broj mobitela korisnika\\ \hline 
					Uloga	& VARCHAR &   definira razinu ovlasti korisnika\\ \hline
					
					
				\end{longtabu}
				
				\noindent \textbf{Poslovnica} \quad Ovaj entitet sadržava sve važne informacije o poslovnici. Sadrži atribute: ID poslovnice i lokaciju poslovnice.
				
				\begin{longtabu} to \textwidth {|X[6, l]|X[6, l]|X[20, l]|}
					
					\hline \multicolumn{3}{|c|}{\textbf{Poslovnica}}	 \\[3pt] \hline
					\endfirsthead
					
					\hline \multicolumn{3}{|c|}{\textbf{Poslovnica}}	 \\[3pt] \hline
					\endhead
					
					\hline 
					\endlastfoot
					
					\cellcolor{LightGreen}ID poslovnice & INT	&  	jedinstveni identifikator poslovnice\\ \hline
					Lokacija	& VARCHAR &   adresa poslovnice\\ \hline
					
					
				\end{longtabu}
				
				\noindent \textbf{Recenzija} \quad Ovaj entitet sadržava sve važne informacije vezane za osvrt korisnika o pruženoj usluzi. Sadrži atribute: ID recenzije,
                ocjenu usluge, opis i korisničko ime korisnika koji ocjenjuje uslugu.
                Ovaj entitet u vezi je \textit{Many-to-One} s entitetom Korisnik preko korisničkog imena.
                
                \begin{longtabu} to \textwidth {|X[6, l]|X[6, l]|X[20, l]|}
					
					\hline \multicolumn{3}{|c|}{\textbf{Recenzija}}	 \\[3pt] \hline
					\endfirsthead
					
					\hline \multicolumn{3}{|c|}{\textbf{Recenzija}}	 \\[3pt] \hline
					\endhead
					
					\hline 
					\endlastfoot
					
					\cellcolor{LightGreen}ID recenzije & INT	&  	jedinstveni identifikator recenzije\\ \hline
					Ocjena	& INT &   ocjena za uslugu\\ \hline
					Opis	& VARCHAR &   komentar na ocjenu usluge\\ \hline
					\cellcolor{LightBlue}Korisničko ime	& VARCHAR &   jedinstveni identifikator korisnika, (korisnik.korisničko ime)	\\ \hline 
					
					
				\end{longtabu}
			
			
			\subsection{Dijagram baze podataka}
				\textit{ U ovom potpoglavlju potrebno je umetnuti dijagram baze podataka. Primarni i strani ključevi moraju biti označeni, a tablice povezane. Bazu podataka je potrebno normalizirati. Podsjetite se kolegija "Baze podataka".}
			
			\eject
			
			
		\section{Dijagram razreda}
		
			\textit{Potrebno je priložiti dijagram razreda s pripadajućim opisom. Zbog preglednosti je moguće dijagram razlomiti na više njih, ali moraju biti grupirani prema sličnim razinama apstrakcije i srodnim funkcionalnostima.}\\
			
			\textbf{\textit{dio 1. revizije}}\\
			
			\textit{Prilikom prve predaje projekta, potrebno je priložiti potpuno razrađen dijagram razreda vezan uz \textbf{generičku funkcionalnost} sustava. Ostale funkcionalnosti trebaju biti idejno razrađene u dijagramu sa sljedećim komponentama: nazivi razreda, nazivi metoda i vrste pristupa metodama (npr. javni, zaštićeni), nazivi atributa razreda, veze i odnosi između razreda.}\\
			
			\textbf{\textit{dio 2. revizije}}\\			
			
			\textit{Prilikom druge predaje projekta dijagram razreda i opisi moraju odgovarati stvarnom stanju implementacije}
			
			
			
			\eject
		
		\section{Dijagram stanja}
			
			
			\textbf{\textit{dio 2. revizije}}\\
			
			\textit{Potrebno je priložiti dijagram stanja i opisati ga. Dovoljan je jedan dijagram stanja koji prikazuje \textbf{značajan dio funkcionalnosti} sustava. Na primjer, stanja korisničkog sučelja i tijek korištenja neke ključne funkcionalnosti jesu značajan dio sustava, a registracija i prijava nisu. }
			
			
			\eject 
		
		\section{Dijagram aktivnosti}
			
			\textbf{\textit{dio 2. revizije}}\\
			
			 \textit{Potrebno je priložiti dijagram aktivnosti s pripadajućim opisom. Dijagram aktivnosti treba prikazivati značajan dio sustava.}
			
			\eject
		\section{Dijagram komponenti}
		
			\textbf{\textit{dio 2. revizije}}\\
		
			 \textit{Potrebno je priložiti dijagram komponenti s pripadajućim opisom. Dijagram komponenti treba prikazivati strukturu cijele aplikacije.}